\documentclass{book}

\usepackage{amsmath, amssymb, geometry, setspace, titlesec}
\usepackage[parfill]{parskip}
\renewcommand{\thesubsection}{\thesection\alph{subsection}}
\onehalfspacing

\titleformat{\subsection}[runin]
  {\normalfont\large\bfseries}{\thesubsection}{1em}{}
\titleformat{\subsubsection}[runin]
  {\normalfont\normalsize\bfseries}{\thesubsubsection}{1em}{}


\title{Math 401: Midterm 2 Study Guide}
\author{Tim Farkas}
\date{Oct 2021}

\begin{document}
\maketitle

\chapter*{Chapter 3: The Real Numbers}

\section*{Definitions \& Theorems}

\subsection*{Theorem 10.6(2) (Principle of Mathematical Induction)}

$P(n)$ is true for all $n > m$, $n, m \in \mathbb{N}$ provded 

(a) $P(m)$ is true, and 

(b) for each $k \in \mathbb{N} \geq m$, if $P(k)$ is true, then $P(k+1)$ is true. 

Note: If $m = 1$, then $P(n)$ is true for all $n \in \mathbb{N}$. 

\subsection*{Theorem 11.9d (Triangle Inequality)}

Let $x, y \in \mathbb{R}$. Then $\left|x + y\right| \leq |x| + |y|$. 

\subsection*{Theorem 12.1}

Let $p$ be a prime number. Then $\sqrt{p}$ is not a rational number. 

\subsection*{Definition 12.2 (Upper / Lower Bound)}

If there exists a real number $m$ such that $m \geq s$ for all $s \in S$, then $m$ is an \emph{upper bound} of $S$, and we say $S$ is "bounded above"

Flip for \emph{lower bound}. 

\subsection*{Definition 12.2 (Maximum / Minimum)}

If there exists an upper (lower) bound $m$ of $S$ that is a member of $S$, then $m$ is the \emph{maximum (minimum)} of $S$. 

\subsection*{Definition 12.5 (Supremum / Infimum)}

An upper (lower) bound $m$ of a non-empty subset of $   \mathbb{R}$ is called the \emph{supremum (infimum)} or \emph{least upper (greatest lower) bound} of $S$ iff

if for all $m' < m$, there exists an $s \in S$ such that $s > m'$. 

Flip for infimum. 

\subsection*{Axiom (Completeness of the Reals)}

Every non-empty subset of $\mathbb{R}$ that is bounded above (below) has a supremum (infimum). 

\subsection*{Theorem 12.9(10) (The Archimedean Property)}

The following are equivalent: 

(a) The set $\mathbb{N}$ of natural numbers is unbounded above in $\mathbb{R}$. 

(b) For each $z \in \mathbb{R}$, there exists an $n \in \mathbb{N}$ such that $n > z$. 

(c) For each $x > 0$ and for each $y \in \mathbb{R}$, there exists an $n \in \mathbb{N}$ such that $nx > y$. 

(d) For each $x > 0$, there exists an $n \in \mathbb{N}$ such that $0 < \frac{1}{n} < x$. 

\subsection*{Theorems 12.12 and 12.13}

If $x$ and $y$ are real numbers and $x < y$, then there exists both an irrational number $i$ such that 
$x < i < y$, and a rational number $r$ such that $x < r < y$. 

\subsection*{Definition 13.3 (Interior Point)}

A point $x$ in $\mathbb{R}$ is an \emph{interior point} of $S$ if there exists a (non-deleted) neighborhood $N$ of $x$ such that $N \subseteq S$. 

\subsection*{Definition 13.3 (Boundary Point)}

A point $x$ in $\mathbb{R}$ is a \emph{boundary point} of $S$ if for every (non-deleted) neighborhood $N$ of $x$, we have $N \cap S \neq \emptyset$ and $N \cap S^c \neq \emptyset$. 

\subsection*{Definition 13.6 (Open / Closed Sets)}

Let $S \in \mathbb{R}$. 

If $    \text{bd} \,S \subseteq S$, then $S$ is \emph{closed}.

If $\text{bd} \,S \subseteq S^c$, then $S$ is \emph{open}. 

\subsection*{Definition 13.14 (Accumulation / Isolated Points)}

A point $x \in \mathbb{R}$ is an \emph{accumulation point} of $S$ if every deleted neighborhood of $x$ contains a point in $S$. 

Otherwise, $x$ is an \emph{isolated point} of $S$. 

\subsection*{Definition 14.1 (Open Cover)}

A family $\mathcal{F}$ of open sets is an \emph{open cover} of the set $S$ if $S \subseteq \bigcup \mathcal{F}$. 

\subsection*{Definition 14.1 (Subcover)}

If $\mathcal{G} \subseteq \mathcal{F}$, and $\mathcal{G}$ is also an open cover of $S$, then $\mathcal{G}$ is a \emph{subcover} of $S$. 

\subsection*{Definition 14.1 (Compact Set)}

The set $S$ is called \emph{compact} if every open cover of $S$ contains a finite subcover. 

\subsection*{Theorem 14.5 (Heine-Borel)}

A subset $S$ of $\mathbb{R}$ is compact iff $S$ is closed and bounded. 

\chapter*{Chapter 4: Sequences}

\section*{Definitions \& Theorems}

\subsection*{Definition 16.2 (Convergence)}

A sequence $s_n$ is said to \emph{converge} to a number $L$ if for every $\epsilon > 0$ there exists a number $N \in  \mathbb{N}$ such that if $n > N$, then $\left|s_n - L\right| < \epsilon$. 

\subsection*{Theorem 16.13}

Every convergent sequence is bounded. 

\subsection*{Theorem 16.14}

If a sequence converges, it's limit is unique. 

\subsection*{Theorem 17.1 (Fundamental Limit Theorems)}

Suppose that $(s_n)$ and $(t_n)$ are convergent sequences with limits $s$ and $t$. Then

(a) $\lim_{n \rightarrow \infty} (s_n + t_n) = s + t$

(b) $\lim_{n \rightarrow \infty} ks_n = ks$ and $\lim_{n \rightarrow \infty} (k + s_n) = k + s$ for any $k \in \mathbb{R}$. 

(c) $\lim_{n \rightarrow \infty} s_nt_n = st$. 

(d) $\lim_{n \rightarrow \infty} \frac{s_n}{t_n} = \frac{s}{t}$. 

\subsection*{Definition 18.1 (Monotone)}

A sequence is \emph{monotone} if either $s_n \leq s_{n+1}$ (increasing) or $s_n \geq s_{n+1}$ (decreasing) for all $n \in \mathbb{N}$. 

\subsection*{Theorem 18.3 (Monotone Convergence Theorem)}

A monotone sequence is convergent iff it is bounded. 

\subsection*{Definition 18.9 (Cauchy Sequence)}

A sequence $(s_n)$ is a \emph{Cauchy sequence} if for all $\epsilon > 0$, there exists a number $N \in \mathbb{N}$ such that if $n, m  > N$, then $\left|s_n - s_m \right| < \epsilon$. 

\subsection*{Theorem 18.12 (Cauchy Convergence Criterion}

A sequence of real numbers is convergent iff it is a Cauchy sequence. 

\subsection*{Definition 19.1 (Subsequence)}

Let $(n_k)$ be a sequence of natural numbers such that $n_1 < n_2 < ... < n_k$. Then $s_{n_k}$ is a \emph{subsequence} of $s_n$. 

\subsection*{Theorem 19.4}

If a sequence $(s_n)$ converges to a real number $s$, then any subsequence of $(s_n)$ converges to $s$. 

\subsection*{Theorem 19.7}

Every bounded sequence has at least one convergent subsequence. 

\subsection*{Definition 19.9 (Limit Superior / Inferior)}

If $S$ is the set of subsequential limits of a sequence $(s_n)$, the \emph{limit superior} $\text{lim sup }s_n = \sup(S)$. The \emph{limit inferior} is defined likewise with the infimum of $S$. 


\end{document}



