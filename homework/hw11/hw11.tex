\documentclass{article}

\usepackage{amsmath, amssymb, geometry, setspace, titlesec}
\usepackage[parfill]{parskip}
\renewcommand{\thesubsection}{\thesection\alph{subsection}}
\onehalfspacing

\titleformat{\subsection}[runin]
  {\normalfont\large\bfseries}{\thesubsection}{1em}{}
\titleformat{\subsubsection}[runin]
  {\normalfont\normalsize\bfseries}{\thesubsubsection}{1em}{}


\title{Math 401: Homework 11}
\author{Tim Farkas}
\date{Nov 2021}

\begin{document}
\maketitle

\section*{Problem 1}

Suppose $f, g: \mathbb{R} \rightarrow \mathbb{R}$ are differentiable at 0 and that $f(0) = g(0) = 0$. Define

\begin{equation*}
        k(x) = \left\{
\begin{array}{ll}
        f\circ g(x) & x<0 \\
        0 & x = 0 \\
        g \circ f(x) & x > 0
\end{array}\right.
\end{equation*}

Show that $k$ is differentiable at 0 and find $k'(0)$. 

Since $f \circ g (x)$ and $g \circ f(x)$ are differentiable at 0, we have that $(f\circ g)'(x) = f'(g(x))g'(x)$ and $(g \circ f)'(x) = g'(f(x))f'(x)$. Evaluating each derivative at $x = 0$ yields

\begin{align*}
        (f \circ g)'(0) &= f'(g(0))g'(0)\\
        &= f'(0)g'(0)
\end{align*}

and 

\begin{align*}
        (g \circ f)'(0) &= g'(f(0))f'(0)\\
        &= f'(0)g'(0)
\end{align*}

Hence we can show that both the left and right limits of $k$ at $x \rightarrow 0$ are equal: 

\begin{align*}
        \lim_{x \rightarrow ^-0} \frac{k(x) - k(0)}{x - 0} &= \frac{(f \circ g)(x) - 0}{x} \\
                                                           &= \frac{(f \circ g)(x) - (f \circ g)(0)}{x - 0} \\
                                                           &= (f \circ g)'(x) \\
                                                           &= f'(0)g'(0)
\end{align*}

and

\begin{align*}
        \lim_{x \rightarrow ^+0} \frac{k(x) - k(0)}{x - 0} &= \frac{(g \circ f)(x) - 0}{x} \\
                                                           &= \frac{(g \circ f)(x) - (g \circ f)(0)}{x - 0} \\
                                                           &= (g \circ f)'(x) \\
                                                           &= f'(0)g'(0)
\end{align*}

Hence $k$ is differentiable at 0 and $k'(0) = f'(0)g'(0)$. 

\section*{Problem 2}

Suppose $f, g: \mathbb{R}\rightarrow \mathbb{R}$, $S$ is a finite set where $S \subseteq (0, \infty)$ and $\forall x \notin S, \, f(x) = g(x)$. 

Show that if $f$ is differentiable at $0$, then $g$ is differentiable at 0 and $f'(0) = g'(0)$. 

Since $S$ is a finite set, $S$ has a minimum and $\min\{S\} > 0$. 

Hence, take $\delta = \min\{S\}$. Then there exists a neighborhood $N(0, \delta)$ wherein $f(x) = g(x)$, and hence $f'(0) = g'(0)$. 

\section*{Problem 3}

Prove that if a polynomial function $p: \mathbb{R} \rightarrow \mathbb{R}$ is divisible by $(x - 2)^3$, then the polynomial $p'(x)$ is divisible by $(x - 2)^2$. 

By definition we can take $p = (x - 2)^3p^*$, where $p^*$ is some polynomial factor of $p$. 

Then we have 
\begin{align*}
        p' &= \frac{d}{dx}(x - 2)^3 p^* \\
           &= 3(x-2)^2p^* + (x-2)^3p'^* & \quad \text{(product rule)}\\
           &= (x-2)^2\left[3p^* + (x-2)p'^* \right] \\
           &= (x-2)p^{**}
\end{align*}

Where $p^{**}$ is a polynomial, since the derivative of a polynomial is a polynomial, and the sum of polynomials is polynomial. 

\section*{Problem 4}

Prove that if $f(x)$ is an even function, then $f'(0) = 0$. 

If $f(x)$ is even, then $f'(x)$ is odd, by proof in recitation. 

If $f'(x)$ is odd, then we have that $f'(0) = f'(-0) = -f(0)$, so $f'(0) = -f'(0)$, and therefore we have that $2f'(0) = 0$, and $f'(0) = 0$. 

\end{document}
