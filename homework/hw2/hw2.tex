\documentclass{article}

\usepackage{amsmath, amssymb, geometry, setspace}
\usepackage[parfill]{parskip}
\renewcommand{\thesubsection}{\thesection\alph{subsection}}
\onehalfspacing

% new commands
\newcommand{\R}{\mathbb{R}}
\newcommand{\Z}{\mathbb{Z}}
\newcommand{\N}{\mathbb{N}}
\newcommand{\Q}{\mathbb{Q}}


\title{Math 401: Homework 2}
\author{Tim Farkas}
\date{Aug 2021}

\begin{document}
\maketitle

\section*{Problem 1}
Define $f$: $(1,4) \rightarrow \mathbb{R}$ by 

\begin{equation*}
  f(x) = \frac{1}{x^2 - 5x + 4}
\end{equation*} 

We can analyze the behavior of $f$ by looking at the denominator only, which we define as $g(x)$.  

$g(x) = x^2 - 5x + 4 = (x - 4)(x - 1)$, hence the function is indeed properly defined on the whole domain $(1,4)$. 

$g'(x) = 2x - 5$, hence there is a single inflection point at 5/2. 

$g''(x) = 2 > 0$, hence the inflection point is a local minimum for $g(x)$, and a local maximum for $f(x)$. 

\subsection*{1a}

$f$ is not one to one, because there is an inflection point within the domain and the inflection point does not occur at either boundary of the domain.

\subsection*{1b}

The inflection point occurs at $x = 5/2$. Because $5/2 > y$, $ \forall y \in (1, 2)$ and $f$ is concave, $f$ is monotonic increasing on $(1,2)$. Hence: 

$f((1, 2)) = (f(1)^+, f(2)^-) = (-\infty, -1/2)$

\subsection*{1c}

Becuase the inflection point occurs at $x = 5/2 \in (1,3)$ and $f$ is concave:

$f(1,3) = (\min(f(1)^+, f(3)), f(5/2)) = (\min(-\infty, -1/2), -1/4) = (-\infty, -1/4)$

\section*{Problem 2}

Define $f: \mathbb{Z} \rightarrow \mathbb{Z}$ by 
\begin{equation*}
  f(x) = 3 \left|x - \frac{1}{3}\right|
\end{equation*}

We rewrite as a piecewise function: 

\begin{equation*}
  f(x) = \left\{
    \begin{array}{ll}
      3x - 1 &\quad x \geq 1/3 \leftrightarrow x \geq 1 \\
      1 - 3x&\quad x < 1/3 \leftrightarrow x \leq 0
  \end{array}\right.
\end{equation*}
\subsection*{2a}

$f$ is one-to-one. 

Proof: 

Both piecewise elements of $f(x)$ are linear in $x$ and hence are monotonic, so both pieces are are themselves one-to-one. 

We have left to show that the ranges of each piece have no common elements, which we prove by contradiction. Take $x < 0 < y$ and $x, y \in \mathbb{Z}$. The conditions under which two functions have the same outcome can be derived:

\begin{align}
  3x - 1 &= 1 - 3y \\
  3x &= 2 - 3y \\
  x &= \frac{2}{3} - y
\end{align}

But then either $x \notin \mathbb{Z}$ or $y \notin \mathbb{Z}$ and the proof is complete. $\square$
\subsection*{2b}

$f$ is not onto, because no negative integers are in the range, but all negative integers are in the codomain. To prove, we factor $f$ into $3$ and $\left| x - 1/3 \right|$ and note that both are non-negative for all $x \in \mathbb{R} \supseteq \mathbb{Z}$. 

\section*{Problem 3}

$f$: $\mathbb{R} \rightarrow \mathbb{R}$ by 

\begin{equation*}
  f(x) = \max(2x, 1)
\end{equation*}

For analysis, we rewrite piecewise: 

\begin{equation*}
  f(x) = \left\{
    \begin{array}{ll}
      1 &\quad x \le 1/2 \\
    2x &\quad x > 1/2
    \end{array}
    \right.
  \end{equation*}

  \subsection*{3a}

  $f^{-1}((0, 1)) = \emptyset$, because $f(x)$ is never less than 1. 

  \subsection*{3b}

$f^{-1}((1, 2)) = (1/2, 1)$. If $f(x) > 1$, as it is for the whole specified subset of the codomain, then $f(x) = 2x$ and $x = y / 2 = f^{-1}(y)$. 
Then, because $f(x) = 2x$ is monotonic increasing, $f^{-1}( (1, 2)) = (f^{-1}(1)^+, f^{-1}(2)^-) = (1/2, 1)$. $\square$ 

  \subsection*{3c}

$f^{-1}((-\infty, 2)) = (-\infty, 1)$. Because the minimum value in the range is 1, $f^{-1}((-\infty, 1) = \emptyset$. 
The remainder of the codomain is $[1, 2)$, which differs from 3b only by the inclusion of 1. But $f^{-1}(\{1\}) = (-\infty, 1/2]$.
Because $(-\infty, 1/2] \cup (1/2, 1) = (-\infty, 1)$, we see that $f^{-1}((-\infty, 2)) = (-\infty, 1)$.

\section*{Problem 4}

Define $f$: $\mathbb{R} \rightarrow \mathbb{R}$ by 

\begin{equation*}
  f(x) = y = \left\{
    \begin{array}{ll}
      x^2 &\quad x \le 0 \\
    x + 1 &\quad x > 0
    \end{array}
    \right.
  \end{equation*}

  Therefore, we have that $f^{-1}(\{y\}) = \{-\sqrt{y}: y \ge 0\} \cup \{y - 1: y > 1\}$. Note the functions defining both sets in this union are monotonic. 

  \subsection*{4a}

  $f^{-1}((0, 1)) = (f^{-1}(1)^+,f^{-1}(0)^-) \cup \emptyset = (-\sqrt{1}, -\sqrt{0}) = (-1, 0)$
  
  \subsection*{4b}

\begin{align*}
  f^{-1}((1, 2)) &= \{\{-\sqrt{y}: y \in (1, 2) \cap [0, \infty)\}, \{y - 1: y \in (1, 2) \cap (1, \infty)\}\} \\
  &= \{(-\sqrt{2}, -1), (0, 1)\} \\
  &= (-\sqrt{2}, -1) \cup (0, 1)
\end{align*}

\subsection*{4c}

$f^{-1}((-\infty, 2)) = (-\sqrt{2}, 1)$. Because the minimum value in the range is 0, $f^{-1}((-\infty, 0) = \emptyset$. 
The remaining range is $[0, 2)$, which differs from the union of ranges from 4a and 4b by inclusion of 0 and 1.

\begin{align*}
  f^{-1}([0, 2)) &= \{\{-\sqrt{y}: y\in [0, 2) \cap [0, \infty) = [0, 2)\}, \{y - 1: y \in [0, 2) \cap (1, \infty) = (1, 2)\}\} \\
&= \{(-\sqrt{2}, 0], (0, 1)\} \\
  &= (-\sqrt{2}, 0] \cup (0, 1) \\
  &= (-\sqrt{2}, 1)
\end{align*}
    
\section*{Problem 5}

If $f$: $\mathbb{R} \rightarrow \mathbb{R}$ and $g$: $\mathbb{R} \rightarrow \mathbb{R}$ are both one-to-one, then the function $h$: $\mathbb{R} \rightarrow \mathbb{R}$ defined by 

\begin{equation*}
  h(x) = \left\{
  \begin{array}{ll}
    f(x) &\quad if x \le 0 \\
    g(x) &\quad if x > 0
  \end{array}
\right.
\end{equation*}

is also one to one. 

This statement is false. Take $f(x) = -x$ and $g(x) = x$. Each are one-to-one. But then $h(x) = |x|$, which is not one-to-one.
\end{document}

