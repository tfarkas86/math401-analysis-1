\documentclass{article}

\usepackage{amsmath, amssymb, geometry, setspace}
\usepackage[parfill]{parskip}
\renewcommand{\thesubsection}{\thesection\alph{subsection}}
\onehalfspacing

\title{Math 401: Homework 1}
\author{Tim Farkas}
\date{Aug 2021}

\begin{document}
\maketitle

\section*{Problem 1}

\subsection*{1a}
Statement: $\forall a \in A, \exists y \in B$ s.t. $ y^2 = x$.

Negation: $\exists a \in A$ s.t. $\forall y \in B$, $y^2 \ne x$

\subsection*{1b}
Statement: $\forall x \in A$, we have $\{x\} \subseteq B$. 

Negation: $\exists x \in A$ s.t. $\{x\} \nsubseteq B$

\subsection*{1c}
Statement: There is an element of A that is also an element of B. 

Negation: $\forall a \in A, a \notin B$. Equivalently, $A \cap B = \emptyset$ 

\section*{Problem 2}
Prove whether the following are true or false. 

\subsection*{2a}
Statement: $\forall x,y \in \mathbb{R}$ we have $x^2 = y^2$

Value: False. 

Proof: Take $x = 0$, $y = 1$. Then $x^2 = 0 \ne 1 = y^2$ $\square$
\subsection*{2b}
Statement: $\forall x \in \mathbb{Z}, \exists y \in \mathbb{R}$ s.t. $x^2 = y^2$

Value: True. 

Proof: Because $\mathbb{Z} \subset \mathbb{R}$, we can always take $x = y$. Then $x^2 = y^2$. $\square$
\subsection*{2c}
Statement: $\forall x \in \mathbb{R}, \exists y \in \mathbb{R}$ s.t. $xy = 1$

Value: False. 

Proof: Take $x = 0$. Then $\forall y$ we have $xy = 0.$ $\square$
\subsection*{2d}
Statement: $\forall x \in \mathbb{R}, \exists y \in \mathbb{R}$ s.t. $xy = 0$

Value: True. 

Proof: Take $y = 0$. Then $\forall x \in \mathbb{R}$ we have $xy = 0$
\subsection*{2e}
Statement: $\exists y \in \mathbb{R}$ s.t. $\forall x \in \mathbb{R}$ we have $xy = 0$. 

Value: True.

Proof: Statement 2e is equivalent to statement 2d. See proof to 2d. 

\section*{Problem 3}
Prove that if $x \in \mathbb{R}$, then $1 < x < 3$ implies $-10 < x^3 - x < 30$.  

Proof: Assume $1 < x < 3$. This implies $1 < x^3 < 27$, and also that  $-3 < -x < -1$. Adding the two implied statements yields $-2 < x^3 - x < 26$, and so we have $1 < x < 3 $ implies $-2 < x^3 - x < 26$. Then, because $-10 < -2 < x^2 - x < 26 < 30$, we see that $1 < x < 3$ implies $-10 < x^3 - x < 30$. $\square$

\newpage
\section*{Problem 4}
For $n \in \mathbb{Z}$, prove that if $n$ is even, then $n^2 = 4k + 4$ for some $k \in \mathbb{Z}$.  

Proof: Assume $n$ is even. Then $\exists q \in \mathbb{Z}$ s.t. $n = 2q$. 
To prove the statement, we substitute $2q$ for $n$ in the consequent. Rearranging a bit first: 

\begin{flalign*}
  k &= (n^2 - 4) / 4 &\\
  k &= ((2q)^2 - 4) / 4 & \\
  k &= (4q^2 - 4) / 4 &\\
  k &= q^2 - 1 \\
\end{flalign*}
Hence, $n^2 = 4k + 4$ whenever $k = q^2 - 1$. $\square$

\section*{Problem 5}
Prove that if $x \in \mathbb{R}$, then $-\frac{7}{16} < x^2 -x < -\frac{5}{16}$ 
implies that $x \notin \left(\frac{1}{10}, \frac{3}{10} \right)$. 

\noindent
Proof: To simplify calculations, we prove the contrapositive. If $x \in \mathbb{R}$, then $\frac{1}{10} < x < \frac{3}{10}$ 
implies $x^2 -x < -\frac{7}{16}$ or $-\frac{5}{16} < x^2 - x$. 

Assume $\frac{1}{10} < x < \frac{3}{10}$. Then $\frac{1}{100} < x^2 < \frac{9}{100}$ and $-\frac{3}{10} < -x < -\frac{1}{10}$. 
Adding the last two inequalities yields $-\frac{29}{100} < x^2 - x < -\frac{1}{100}$. 

Hence, if $\frac{1}{10} < x < \frac{3}{10}$, 
then $-\frac{29}{100} < x^2 - x$. To finish the proof, we show that 
$-\frac{5}{16} = -\frac{500}{1600} < -\frac{464}{1600} = -\frac{29}{100} < x^2 - x$, satisfying the condition. $\square$ 

\end{document}

