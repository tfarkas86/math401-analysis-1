\documentclass{article}

\usepackage{amsmath, amssymb, geometry, setspace, titlesec}
\usepackage[parfill]{parskip}
\renewcommand{\thesubsection}{\thesection\alph{subsection}}
\onehalfspacing

\titleformat{\subsection}[runin]
  {\normalfont\large\bfseries}{\thesubsection}{1em}{}
\titleformat{\subsubsection}[runin]
  {\normalfont\normalsize\bfseries}{\thesubsubsection}{1em}{}


\title{Math 401: Homework 8}
\author{Tim Farkas}
\date{Oct 2021}

\begin{document}
\maketitle

\section*{Problem 1}

Suppose $f: \mathbb{R} \rightarrow \mathbb{R}$, and $g: \mathbb{R} \rightarrow \mathbb{R}$ are functions. Suppose $c$ and $L$ are real numbers. Define $h: \mathbb{R} \rightarrow \mathbb{R}$ by 

\begin{equation*}
        h(x) = \left\{
                \begin{array}{ll}
                        f(x) & \quad \text{if } x < c \\
                        g(g) & \quad \text{if } x \geq c
                \end{array}	
                \right.
\end{equation*}

Show that if $\lim_{x \rightarrow c} f(x) = L$ and $\lim_{x \rightarrow c} g(x) = L$, then $\lim_{x \rightarrow c} h(x) = L$. 

Proof: 

Since $\lim_{x \rightarrow c} f(x) = L$, we have that for a given $\epsilon$, there exists a $\delta_f$ such that if $\left|x - c\right| < \delta_f$, then $\left|f(x) - L\right| < \epsilon$.

Likewise, since $\lim_{x \rightarrow c} g(x) = L$, we have that for the same $\epsilon$, there exists a $\delta_g$ such that if $\left|x - c\right| < \delta_g$, then $\left|g(x) - L\right| < \epsilon$.

Take $\delta = \min\{\delta_f, \delta_g\}$. Then if $\left|x - c\right| < \delta$, we have that $\left|x - c\right| < \delta_f$ and $\left|x-c\right| < \delta_g$, and $\left|f(x) - L\right| < \epsilon$ and $\left|g(x) - L\right| < \epsilon$.
Therefore we have that $\left|h(x) - L\right| < \epsilon$, and $\lim_{x \rightarrow c} h(x) = L$. $\square$.  

\section*{Problem 2}

Define $f: \mathbb{R} \rightarrow \mathbb{R}$ by 
\begin{equation*}
        f(x) = \left\{
\begin{array}{ll}
        2x & \quad \text{if $x$ is irrational} \\
        3x & \quad \text{if $x$ is rational}
\end{array}\right.
\end{equation*}

Show that $\lim_{x \rightarrow 0} f(x) = 0$. 

Proof: Because between any two real numbers $x < y$ there exists an irrational number $r$, it suffices to show that $\lim_{x \rightarrow 0} 2x = 0$. 

Take $\delta = \frac{\epsilon}{3}$. Then we have that $\left|f(x) - 0\right| = 3x < \epsilon$  whenever $\left|x - 0\right| = x < \delta = \frac{\epsilon}{3}$. 

\section*{Problem 3}

Suppose $f: \mathbb{R} \rightarrow \mathbb{R}$ is a function and define $g: \mathbb{R} \rightarrow \mathbb{R}$ by 

\begin{equation*}
        g(x) = \left\{
\begin{array}{ll}
        f(x - 1) & \quad \text{if $x < 5$} \\
        f(4) & \quad \text{if $x \geq 5$}
\end{array}\right.
\end{equation*}

Show that if $\lim_{x \rightarrow 1} f(x)  = L$, then $\lim_{x \rightarrow 2} g(x) = L$. 

Since $\lim_{x \rightarrow 1} f(x) = L$ we have a $\delta > 0$ such that $\left|f(x) - L \right| < \epsilon$ for any $\epsilon$ whenever $\left|x - 1\right| < \delta$. Therefore, we also have that $\left|f(x - 1) - L\right| < \epsilon$ whenever $\left|(x - 1) - 1\right| = \left|x - 2\right| < \delta$. We are only concerned with cases when $x < 5$ so $g(x) = f(x - 1)$, hence $\left|f(x - 1) - L\right| = \left|g(x) - L\right| < \epsilon$, and $\lim_{x \rightarrow 2} g(x) = L$. 
\end{document}
