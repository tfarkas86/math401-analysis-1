\documentclass{article}

\usepackage{amsmath, amssymb, geometry, setspace, titlesec}
\usepackage[parfill]{parskip}
\renewcommand{\thesubsection}{\thesection\alph{subsection}}
\onehalfspacing

\titleformat{\subsection}[runin]
  {\normalfont\large\bfseries}{\thesubsection}{1em}{}
\titleformat{\subsubsection}[runin]
  {\normalfont\normalsize\bfseries}{\thesubsubsection}{1em}{}


\title{Math 401: Homework 9}
\author{Tim Farkas}
\date{Oct 2021}

\begin{document}
\maketitle

\section*{Problem 1}

Define $f: \mathbb{R} \rightarrow \mathbb{R}$ by 

\begin{equation*}
        f(x) = \left\{
\begin{array}{ll}
        x &\quad \text{$x$ is rational} \\
        2\left|x\right| &\quad \text{$x$ is irrational}
\end{array}\right.
\end{equation*}

Prove that $f$ is continuous at $c = 0$. 

Proof. Take $\delta = \epsilon/2$. There are two cases. 

If $x$ is rational, then we have $\left|x - c\right| = \left|x - 0\right| = \left|x\right| < \delta = \epsilon/2$. 
This implies that $\left|f(x) - f(0)\right| = \left|x - 0\right| = \left|x\right| < \delta = \epsilon/2 < \epsilon$. 

If $x$ is irrational, then we have $\left|f(x) - f(0)\right| = \left|2\left|x\right|\right| = 2\left|x\right| < 2\delta = \epsilon$. 

Hence in both cases, $\left|x - 0\right| < \delta = \epsilon/2$ implies $\left|f(x) - 0\right| < \epsilon$, proving $f$ is continous at $c = 0$. 

\section*{Problem 2}

Given $f$ as in Problem 1, show that $f$ is not continuous at $c = 2$. 

We prove by showing that there exists a sequence $(a_n)$ that converges to $c = 2$, but which $f((a_n))$ does not converge to $f(2)$. 

Proof. Take $(a_n) = 2 - \frac{1}{\sqrt{n}}$, where $n$ is a prime number. Then we have that $a_n$ is irrational for all $n$, and hence $f((a_n)) = 2\left|2 - \frac{1}{\sqrt{n}}\right| = \left|4 - \frac{2}{\sqrt{n}}\right|$. 
Then $(a_n)$ converges to 2, but $f((a_n))$ converges to 4, but $f(2) = 2$ because 2 is rational, so $\lim_{n \rightarrow \infty} f((a_n)) \neq f(2)$, and $f$ is not continous at $c = 2$. 

\section*{Problem 3}

Suppose $f, g, h: \mathbb{R} \rightarrow \mathbb{R}$, $c \in \mathbb{R}$, and $f(x) \leq g(x) \leq h(x)$ for all $x \in \mathbb{R}$. 

Show that if $f(c) = h(c)$ and $f$ and $h$ are continuous at $c$, that $g$ is also continous at $c$. 

Proof: Since $f(x) \leq g(x) \leq h(x)$ for all $x$ (including $c$), we have that $f(c) = h(c)$ implies that $f(c) = g(c) = h(c)$. 

Now, since $f$ is continous at $c$, we have that $\forall \epsilon > 0$, $\exists \delta_1 > 0$, such that $\left|x - c\right| < \delta_1$ implies that $\left|f(x) - f(c)\right| < \epsilon \Leftrightarrow f(c) - \epsilon < f(x) < f(c) + \epsilon$. 

Likewise for $h$, since $h$ is continous at $c$, we have that $\forall \epsilon > 0$, $\exists \delta_2 > 0$, such that $\left|x - c\right| < \delta_1$ implies that $\left|h(x) - h(c)\right| < \epsilon \Leftrightarrow h(c) - \epsilon < h(x) < h(c) + \epsilon$.

Take $\delta^* = \min\{\delta_1, \delta_2\}$. First, if $\left|x - c\right| < \delta^*$, we have that both $f$ and $h$ are continuous, since $\delta* \leq \delta_1$ and $\delta^* \leq \delta_2$. 

Then, since $f(x) \leq g(x) \leq h(x)$, we have that $f(c) - \epsilon < f(x) < g(x) < h(x) < h(c) + \epsilon$, and since $f(c) = g(c) = h(c)$, we have that $g(c) - \epsilon < g(x) < g(c) + \epsilon \Leftrightarrow \left|g(x) - g(c)\right| < \epsilon$, proving that $g$ is continuous at $c$. 

\section*{Problem 4}

Let

\begin{equation*}
        D = \{0\} \cup \left\{\frac{1}{n}\middle | n \in \mathbb{N}\right\}
\end{equation*}

and suppose $f: D \rightarrow \mathbb{R}$ is a function. Let $(a_n)$ be the sequence defined 

\begin{equation*}
        a_n = f\left(\frac{1}{n}\right)
\end{equation*}

Show that if $\lim_{n \rightarrow \infty} a_n = f(0)$, then $f$ is continuous at 0. 

Proof. Take $x_n = \frac{1}{n}$. Then we have that $(x_n)$ converges to 0, and we have been given that $a_n = f((x_n))$ converges to $f(0)$. Hence, by Theorem 21.2b, $f$ is continuous at 0. 



\end{document}
